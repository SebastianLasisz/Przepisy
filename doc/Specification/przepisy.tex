\documentclass{article}
\usepackage{blindtext}
\usepackage[utf8]{inputenc}
\usepackage[T1]{fontenc}
\usepackage[english,polish]{babel}
\usepackage{polski}
\usepackage{graphicx}
\usepackage{float}

\usepackage{color}
\usepackage{xcolor}
\usepackage{listings}

\usepackage{caption}
\DeclareCaptionFont{white}{\color{white}}
\DeclareCaptionFormat{listing}{\colorbox{gray}{\parbox{\textwidth}{#1#2#3}}}
\captionsetup[lstlisting]{format=listing,labelfont=white,textfont=white}

\graphicspath{ {images/} }
\setlength{\topmargin}{0in}
\setlength{\textheight}{9in}
\setlength{\oddsidemargin}{.125in}
\setlength{\textwidth}{6.25in}

\setlength{\topmargin}{-.5in}
\setlength{\textheight}{9in}
\setlength{\oddsidemargin}{.125in}
\setlength{\textwidth}{6.25in}

\title{Integracja systemów informacyjnych}
\author{
	Sebastian Łasisz
	\and
	Dominik Janusiewicz}
\date{\today}

\begin{document}
\maketitle

\section{Opis funkcjonalno"sci projektu}

	Projekt docelowo b"edzie sk"lada"l si"e z czterech element"ow: dw"och zewn"etrznych aplikacji udost"epniaj"acych API (Edaman, Trello) oraz dw"och 	aplikacji, kt"ore b"ed"a wspiera"c proces tworzenia przepisów kuchennych. 

\subsection{Edaman}
	Edamam jest to system, kt"ory umo\.zliwia wyszukiwanie przepis"ow posiłk"ow oraz umo\.zliwa analiz"e składnik"ow w czasie rzeczywistym. W ramach działania serwisu zostały udost"epnione cztery r"o\.zne API:
	\begin{itemize}
		\item Recipe Analysis and Nurtrition API
		\item Nutrition Data API
		\item Diet Recommendations API
		\item Recipe Search API
	\end{itemize}
	
	\begin{lstlisting}[label=edaman,caption=Edaman API Example,breaklines=true]
{
  "title": "Fresh Ham Roasted With Rye Bread and Dried Fruit Stuffing",
  "prep": "1. Have your butcher bone and butterfly the ham and score the fat in a diamond pattern. ...",
  "yield": "About 15 servings",
  "ingr": [
    "1 fresh ham, about 18 pounds, prepared by your butcher (See Step 1)",
    "7 cloves garlic, minced",
    "1 tablespoon caraway seeds, crushed",
    "4 teaspoons salt",
    "Freshly ground pepper to taste",
    "1 teaspoon olive oil",
    "1 medium onion, peeled and chopped",
    "3 cups sourdough rye bread, cut into 1/2-inch cubes",
    "1 1/4 cups coarsely chopped pitted prunes",
    "1 1/4 cups coarsely chopped dried apricots",
    "1 large tart apple, peeled, cored and cut into 1/2-inch cubes",
    "2 teaspoons chopped fresh rosemary",
    "1 egg, lightly beaten",
    "1 cup chicken broth, homemade or low-sodium canned"
  ]
}
	\end{lstlisting}

\subsection{Trello}
	Trello jest to system, kt"ory umo\.zliwa tworzenie tablic wypełnion"a kartami. Ka\.zda karta mo\.ze by"c notatk"a/list"a zadań/etc... Trello API umo\.zliwa tworzenie nowych tablic/notatek jak i dodawanie nowych, bad"z modyfikowanie czy usuwanie starych. Przykładowe zapytanie zwr"ocenia informacji o karcie:

	\begin{lstlisting}[label=trello,caption=Trello API Example,breaklines=true]
{
    "id": "4eea503d91e31d174600008f",
    "name": "Learn about the Trello API",
    "idList": "4eea4ffc91e31d174600004b"
}
	\end{lstlisting}

\subsection{Aplikacja Przepisy}
	Aplikacja b"edzie pozwala"la u"zytkownikowi na tworzenie dowolnych przepis"ow. Dane przepisy b"ed"a mog"ly by"c prywatne b"ad"z publiczne. Wszystkie przepisy publiczne b"ed"a dost"epne w API kt"ore zostanie udost"epnione. Przyk"ladowe zapytania do API mog"a obejmowa"c:
	\begin{itemize}
		\item Wszystkie przepisy
		\item Przepisy zawieraj"ace dany sk"ladnik
	\end{itemize}
	Ponadto wszystkie przepisy b"ed"a mia"ly informacj"e o alergenach oraz o autorze danego przepisu. Przyk"ladowy wynik zapytania do API mo\.ze zwr"oci"c nast"epuj"acy wynik:

	\begin{lstlisting}[label=example-xml,caption=Przyk"ladowe API w XML,language=XML,breaklines=true]
<?xml version="1.0" encoding="UTF-8" ?>
<przepisy>
    <przepis>
        <id></id>
        <nazwa></nazwa>
        <autor></autor>
        <skladnik>
            <id></id>
            <ilosc></ilosc>
            <nazwa></nazwa>
            <alergen></alergen>
        </skladnik>
    </przepis>
</przepisy>
	\end{lstlisting}

\subsection{Aplikacja mobilna}
	Aplikacja mobilna b"edzie si"e komunikowa"la z aplikacją gł"own"a zar"owno przez kolejk"e komunikat"ow jak i przez udost"epnione API. Zadaniem aplikacji moblinej b"edzie mo"zliwo"s"c oceny danego przepisu (swojego b"ad"z udost"epnionego przez innego u\.zytkownika). Ponadto b"edzie mo\.zliwo"s"c modyfikowania{\textbackslash}usuwania{\textbackslash}dodawania przepis"ow. Ponadto aplikacja mobilna b"edzie mia"la mo"zliwo"s"c skanowania kod"ow kreskowych.

	Do komunikacji mi"edzy zaimplementowanymi aplikacjami zostanie wykorzystany ZeroMQ, jako kolejka komunikat"ow, oraz JSON i XML wraz z XSLT jako API jednego z serwis"ow. 

\end{document}